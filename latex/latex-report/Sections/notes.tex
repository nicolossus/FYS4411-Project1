

%----------------------------------------------------------------
\subsection*{Variational Monte Carlo}
%---------------------------------------------------------------- 

\textbf{Mathemical foundation of the Monte Carlo methods.}

The standard Monte Carlo approximation of $\mathbb{E}[f(X)]$, for some random variable $X:\Omega\to\R$, and a continous function $f:\R\to\R$ is
\begin{equation}
    E_M[f](\omega)= \frac{1}{M}\sum_{m=1}^Mf(X_m(\omega)), \quad \omega\in\Omega
\end{equation}
where $M\in\mathbb{N}$, $X_1, \dots, X_m$ are independent and identically distributed random varibles distributed with $X$. For definitions of indepent and identically distributed random variables, see appendix \ref{app:stochastic_maths}. We denote the standard deviation of the stochastic variable $f(X)$ as $\sigma_f$. The mean square error of the Monte Carlo approximation can be shown (shown in \ref{app:MC_error}) to be 
\begin{equation}
    \mathcal{E}_M(f) = \frac{\sigma_f}{\sqrt{M}}. 
\end{equation}
As $M\to\infty$, the error $\epsilon_M$ will therefore approach zero.

\textbf{The Variational Principle.}

Given a Hamiltonian $H$ and a trial wave function $\Psi_T$, the variational principle states that the expectation value $\expval{H}$, defined through
\begin{equation}
    \mathbb{E}[H]=\expval{H}= \frac{\int d\mathbf{R}\Psi_T^*(\mathbf{R})H(\mathbf{R})\Psi_T(\mathbf{R})}{\int d\mathbf{R}\Psi_T^*(\mathbf{R})\Psi_T(\mathbf{R})}, 
\end{equation}
is an upper bound tho the ground state energy $E_0$ of the Hamiltonian $\hat{H}$, that is 
\begin{equation}
    E_0 \leq \expval{H}.
\end{equation}
The eigenstates, $\psi_i$, of the Hamiltonian,
\begin{equation}
    H\psi_i(\mathbf{R}) = E_i\psi_i(\mathbf{R}), 
\end{equation}
form a complete set. The trial wave function can therefore be expanded in terms of them,
\begin{equation}
    \Psi_T(\mathbf{R}) = \sum_i a_i\Psi_i(\mathbf{R}), 
\end{equation}
and assuming the set of eigenfunctions to be normalized we obtain
\begin{equation}
    \frac{\sum_{nm}a_m^*a_n\int d\mathbf{R}\Psi_m^*(\mathbf{R})H(\mathbf{R})\Psi_n(\mathbf{R})}{\sum_{nm}a_m^*a_n\int d\mathbf{R}\Psi_m^*(\mathbf{R})\Psi_n(\mathbf{R})} = \frac{\sum_n a_n^2E_n}{\sum_n a_n^2} \geq E_0, 
\end{equation}
and the equality holds only if $\Psi_T = \psi_0$. Thus, the variational principle states that the lowest expectation value is our best approximation to the ground state. We utilise this by making a wave function that has a number of variational parameters, and search for a minimum in the space of the variational parameters. Note also that the moments of the Hamiltonian becomes
\begin{equation}
    \expval{H^N} = \frac{\int d\mathbf{R}\Psi_T^*(\mathbf{R}, \mathbf{\alpha})H^N\Psi_T(\mathbf{R},\mathbf{\alpha})}{\int d\mathbf{R}\Psi_T^*(\mathbf{R}, \mathbf{\alpha})\Psi_T(\mathbf{R}, \mathbf{\alpha})} = E_0^N
\end{equation}
when $\Psi_T=\psi_0$. The variance, 
\begin{equation}
    \text{Var}[E] = \expval{H^2}-\expval{H}^2, 
\end{equation}
is therefore zero when the ground state is found. Variation is then performed by minimizing both energy and variance. 
