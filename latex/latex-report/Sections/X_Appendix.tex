%================================================================
\section{Appendix}\label{sec:Appendix A}
%================================================================

%----------------------------------------------------------------
\subsection{Mathematical foundation of random variables}\label{app:stochastic_maths}
%---------------------------------------------------------------- 
Fix a set of $\Omega$, the set of outcomes or seeds. We call this set for the sample space, or the set of all possible outcomes. A random variable is a function $X:\Omega \to\R$ (or $X:\Omega\to\R^d$). If another function $F: \R\to\R$, then $F(X)$ is a new random variable. The expectation is a function $\mathbb{E}$ such that, when $X, Y$ are random variables
\begin{enumerate}
    \item $\mathbb{E}[\alpha X + \beta Y] = \alpha\mathbb{E}[X]+\beta\mathbb{E}[Y] \quad \forall \alpha, \beta \in\R$.
    \item $\mathbb{E}[X]=\alpha \text{ if } X(\omega) = \alpha\quad \forall\omega\in\Omega. $
\end{enumerate}

\textbf{Independent random variables.}

Two random variables $X, Y: \Omega\to\R$ are independent if 
\begin{equation*}
    \mathbb{E}[f(X)g(Y)] = \mathbb{E}[f(X)]\mathbb{E}[g(Y)], 
\end{equation*}
for any continuous $f, g: \R\to\R$. 

\textbf{Identically distributed random variables.}

Two random variables $X,Y:\Omega\to\R$ are identically distributed if $\mathbb{E}[f(X)] = \mathbb{E}[f(Y)]$ for any continuous $f:\R\to\R$. 

\textbf{Uniformly distributed random variables.}

A random variable $X:\Omega\to\R$ is uniformly distributed in $[a,b]$ if 
\begin{equation*}
    \begin{split}
        \mathbb{E}[f(X)] &= \frac{1}{b-a}\int_a^bf(y)dy \\ 
        &\Leftrightarrow \int_a^b f(y)dy = (b-a)\mathbb{E}[f(X)], 
    \end{split}
\end{equation*}
for any continuous $f:\R\to\R$.

%----------------------------------------------------------------
\subsection{Error in standard Monte Carlo approximation}\label{app:MC_error}
%---------------------------------------------------------------- 
The mean square error $\epsilon_M$ of the Monte Carlo approximation can be written as 
\begin{equation}
    \mathcal{E}_M(\mathbb{E}[f(X)],E_M[f]) = \sqrt{\mathbb{E}[(\mathbb{E}[f(X)]-E_M[f])^2]}. 
\end{equation}
First we recognize the fact that the expecation value of the Monte Carlo approximation is equal to the true expectation value,  $\mathbb{E}[f(X_m)]=\mathbb{E}[f(X)]\quad \forall m\in[1, M]\in\mathbb{N}$, which must be true as $X$ and $X_m$ are identically distributed. We denote the true expecation value, $\mathbb{E}[f(X)]=\mu$. 
We square $\mathcal{E}_M$ 
\begin{equation*}
    \begin{split}
        \mathcal{E}_M^2 &= \mathbb{E}[(\mathbb{E}[f(X)]-E_M[f])^2] \\
        &= \mathbb{E}[\mathbb{E}[f(X)]-2\mathbb{E}[f(X)]E_M[f]+E_M[f]^2] \\
        &= \mathbb{E}[f(X)]-2\mathbb{E}[f(X)]\mathbb{E}[E_M[f]]+\mathbb{E}[E_M[f]^2] \\
        &= \mu^2-2\mu\frac{1}{M}\sum_{m=1}^M\mathbb{E}[f(X_m)] + \frac{1}{M^2}\sum_{m=1}^M\sum_{n=1}^M\mathbb{E}[f(X_m)f(X_n)]. \\
    \end{split}
\end{equation*}
We acknowledge the fact that $f(X_m)$ and $f(X_n)$ are indepent variables, and thus
\begin{equation*}
    \begin{split}
        \mathcal{E}_M^2 &= \mu^2-2\mu\frac{1}{M}\sum_{m=1}^M\mathbb{E}[f(X_m)] + \frac{1}{M^2}\sum_{m=1}^M\sum_{n=1}^N\mathbb{E}[f(X_m)]\mathbb{E}[f(X_n)] \\
        &= \mu^2-2\mu\frac{M\mu}{M} + \frac{1}{M^2}\sum_{m=1}^M(\sum_{m\neq n}\mu^2 + \mathbb{E}[f(X)^2]) \\
        &= -\mu^2+\frac{1}{M^2}((M^2-M)\mu^2 + M\mathbb{E}[f(X)^2]) \\
        &= -\frac{\mu^2}{M} + \frac{1}{M}\mathbb{E}[f(X)^2] \\
        &= \frac{\mathbb{E}[f(X)^2]-\mathbb{E}[f(X)]^2}{M} = \frac{\text{Var}[f(X)]}{M} \\
    \end{split}
\end{equation*}
which leads to 
\begin{equation*}
    \mathcal{E}_M = \frac{\sqrt{\text{Var}[f(X)]}}{\sqrt{M}}.
\end{equation*}
%------------------------------------------------------------------
\subsection{Derivation of characteristic length of spherical trap.}
%------------------------------------------------------------------
A single-dimensional harmonic oscillator potential has the Hamiltonian,
\begin{equation*}
    H_{1D-ho} = -\frac{\hbar^2}{2m}\dv[2]{x} + \frac{m\omega^2_{ho}x^2}{2}, 
\end{equation*}
where $x$ is the position in 1D-space, $m$ is the weight of the particle in the potential, $\hbar$ is the reduced Planck's constant and $\omega_{ho}$ is the angular frequency of the oscillator potential. The ground state energy of this single-dimensional particle in this harmonic potential is $E_0=\frac{\hbar\omega_{ho}}{2}$. Thus the analytical expectation value of 



\subsection{Calculations regarding derivatives of the wave function}
In this section we swap the notation for expectation value $\mathbb{E}[f]\to\expval{f}$. 
We have a trial wave function, $\Psi_T(\mathbf{R}; \mathbf{\alpha})$, and we want to solve the optimization problem 
\begin{equation*}
    \frac{d\expval{E_L(\mathbf{R}; \mathbf{\alpha})}}{d\mathbf{\alpha}} = 0. 
\end{equation*}
In quantum mechanics, given a wave function, $\Psi(\mathbf{R})$, the expectation value of a general operator $\hat{Q}$ is defined as 
\begin{equation*}
    \expval{Q} = \frac{\int_{\mathbf{R}\in\mathbb{D}}\Psi^*(\mathbf{R})\hat{Q}\Psi(\mathbf{R})d\mathbf{R}}{\int_{\mathbf{R}\in\mathbb{D}}\Psi^*\Psi d\mathbf{R}}, 
\end{equation*}
in the case where the general wave function $\Psi$ does not have to be normalized. 
This is more readable using Dirac notation, where the equation above is rewritten as 
\begin{equation*}
    \expval{Q} = \frac{\bra{\Psi(\mathbf{R}; \mathbf{\alpha})}\hat{Q}\ket{\Psi(\mathbf{R};\mathbf{\alpha})}}{\braket{\Psi_T(\mathbf{R}; \mathbf{\alpha})}{\Psi_T(\mathbf{R}; \mathbf{\alpha})}}, 
\end{equation*}
which we will simplify further by $\Psi_T(\mathbf{R}; \mathbf{\alpha})\to\Psi$ 
\begin{equation}\label{eq:expectation_value}
    \expval{Q} = \frac{\bra{\Psi}\hat{Q}\ket{\Psi}}{\braket{\Psi}}.
\end{equation}
Now, the local energy, $E_L$ is defined as 
\begin{equation*}
    E_L(\mathbf{R}; \mathbf{\alpha}) = \frac{1}{\Psi_T(\mathbf{R}; \mathbf{\alpha})}\hat{H}\Psi_T(\mathbf{R}; \mathbf{\alpha}),  
\end{equation*}
and inserting this in \ref{eq:expectation_value}, we get
\begin{equation*}
    \expval{E_L} = \frac{\bra{\Psi}\frac{1}{\Psi}\hat{H}\Psi\ket{\Psi}}{\braket{\Psi}}=\frac{\bra{\Psi}\hat{H}\ket{\Psi}}{\braket{\Psi}}.  
\end{equation*}
We want to evaluate the $\frac{d\expval{E_L}}{d\mathbf{\alpha}}$, and we get 
\begin{equation*}
    \begin{split}
        \frac{d\expval{E_L}}{d\alpha} =& \frac{\braket{\Psi}(\bra{\frac{d\Psi}{d\mathbf{\alpha}}}\hat{H}\ket{\Psi}+\bra{\Psi}\frac{d\hat{H}}{d\mathbf{\alpha}}\ket{\Psi}+\bra{\Psi}\hat{H}\ket{\frac{d\Psi}{d\mathbf{\alpha}}})}{\braket{\Psi}^2}\\
        &-\frac{\bra{\Psi}\hat{H}\ket{\Psi}(\bra{\frac{d\Psi}{d\mathbf{\alpha}}}\ket{\Psi}+\bra{\Psi}\ket{\frac{d\Psi}{d\mathbf{\alpha}}})}{\braket{\Psi}^2}
    \end{split}
\end{equation*}

\begin{equation*}
    \begin{split}
        \frac{d\expval{E_L}}{d\alpha} =& \frac{\bra{\frac{d\Psi}{d\alpha}}\hat{H}\ket{\Psi} + \bra{\Psi}\hat{H}\ket{\frac{d\Psi}{d\alpha}}}{\braket{\Psi}}  \\
        &- \frac{\bra{\Psi}\hat{H}\ket{\Psi}(\bra{\frac{d\Psi}{d\alpha}}\ket{\Psi}+\bra{\Psi}\ket{\frac{d\Psi}{d\alpha}}}{\braket{\Psi}^2} \\
        =& \frac{\bra{\frac{d\Psi}{d\alpha}}\hat{H}\ket{\Psi} + \bra{\Psi}\hat{H}\ket{\frac{d\Psi}{d\alpha}}}{\braket{\Psi}} \\
        &- \expval{E_L}\frac{\bra{\frac{d\Psi}{d\alpha}}\ket{\Psi}+\bra{\Psi}\ket{\frac{d\Psi}{d\alpha}}}{\braket{\Psi}}
    \end{split}
\end{equation*}
assuming hermiticity in the Hamiltonian, we can make further simplifications 
\begin{equation}\label{eq:possible_solution}
    \begin{split}
        \frac{d\expval{E_L}}{d\alpha} = 2\expval{E_L\frac{d}{d\alpha}}-2\expval{E_L}\expval{\frac{d}{d\alpha}}
    \end{split}
\end{equation}



Second way: 
\begin{equation*}
    \expval{E_L} = \int_{\mathbf{R}\in\mathbb{D}}\frac{1}{\Psi[\alpha]}\hat{H}\Psi[\alpha]d\mathbf{R}. 
\end{equation*}
% forsøk på utledning. 
\begin{equation*}
    \begin{split}
        \frac{d\expval{E_L[\alpha]}}{d\alpha} =& \int d\mathbf{R}(\frac{d}{d\alpha}\frac{1}{\Psi[\alpha]})\hat{H}\Psi[\alpha] \\
        &+ \int d\mathbf{R}\frac{1}{\Psi[\alpha]}\frac{d\hat{H}}{d\alpha}\Psi[\alpha] \\
        &+ \int d\mathbf{R}\frac{1}{\Psi[\alpha]}\hat{H}\frac{d\Psi[\alpha]}{d\alpha} \\
        =& \int d\mathbf{R}\frac{\frac{d(\hat{H}\Psi[\alpha])}{d\alpha}-\frac{d\Psi[\alpha]}{d\alpha}\hat{H}\Psi[\alpha]}{\Psi[\alpha]^2} \\
        =& \int d\mathbf{R}\frac{\hat{H}\frac{d\Psi[\alpha]}{d\alpha}}{\Psi[\alpha]^2} - \expval{E_L}\int d\mathbf{R}\frac{\frac{d\Psi[\alpha]}{d\alpha}}{\Psi[\alpha]} \\
        =& \frac{d}{d\alpha}\bra{\frac{1}{\Psi[\alpha]}}\hat{H}\ket{\Psi[\alpha]} \\
        =& \bra{\frac{d}{d\alpha}\frac{1}{\Psi}}\hat{H}\ket{\Psi} + \bra{\frac{1}{\Psi}}\frac{d\hat{H}}{d\alpha}\ket{\Psi} + \bra{\frac{1}{\Psi}}\hat{H}\ket{\frac{d\Psi}{d\alpha}} \\
        =& -\bra{\frac{d\Psi}{d\alpha}\frac{1}{\Psi^2}}\hat{H}\ket{\Psi} + \bra{\frac{1}{\Psi}}\hat{H}\ket{\frac{d\Psi}{d\alpha}} \\
        =& -\expval{E_L}\expval{\frac{dln\Psi}{d\alpha}} + \expval{E_L\frac{dln\Psi}{d\alpha}} \\
        =& -\bra{\frac{1}{\Psi}}\hat{H}\ket{\Psi}\expval{\frac{\frac{d\Psi}{d\alpha}}{\Psi}}
    \end{split}
\end{equation*}

% forsøk 3
Third way: 
\begin{equation*}
    \begin{split}
        \expval{E_L} =& \int_{\mathbf{R}\in\mathbb{D}}d\mathbf{R}\Psi^*\frac{1}{\Psi}\hat{H}\Psi\Psi \\
        =& \bra{\frac{\Psi}{\Psi^*}}\hat{H}\ket{\Psi^2} 
    \end{split}
\end{equation*}

\begin{equation*}
    \expval{E_L} = \int d\mathbf{R}\frac{\Psi^*}{\Psi}\hat{H}\Psi^2
\end{equation*}

\begin{equation*}
    \begin{split}
        \frac{d\expval{E_L}}{d\alpha} =& \int \frac{d\frac{\Psi^*}{\Psi}}{d\alpha}\hat{H}\Psi^2 \\
        &+\int \frac{\Psi^*}{\Psi}\hat{H}\frac{d\Psi^2}{d\alpha} \\
        =& \int \frac{\frac{d\Psi^*}{d\alpha}\Psi - \Psi^*\frac{d\Psi}{d\alpha}}{\Psi^2}\hat{H}\Psi^2  \\
        &+ 2\int \frac{\Psi^*}{\Psi}\hat{H}\Psi\frac{d\Psi}{d\alpha} \\
        =& \int \frac{d\Psi^*}{\Psi d\alpha}\hat{H}\Psi^2 \\
        &- \int \frac{\Psi^*\frac{d\Psi}{d\alpha}}{\Psi^2}\hat{H}\Psi^2 \\
        &+ 2\int \frac{\Psi^*}{\Psi}\hat{H}\Psi\frac{d\Psi}{d\alpha} \\
    \end{split}
\end{equation*}

\begin{equation*}
    \begin{split}
        \frac{d\expval{E_L}}{d\alpha} =& \bra{\frac{d\Psi^*}{d\alpha}\frac{1}{\Psi}}\hat{H}\ket{\Psi^2} \\
        &- \bra{\frac{\Psi^*}{\Psi}}\hat{H}\ket{\Psi^2}\bra{\frac{\Psi^*}{\Psi}}\frac{\frac{d\Psi}{d\alpha}}{\Psi}\ket{\Psi^2} \\
        &+ 2\bra{\frac{\Psi^*}{\Psi}}\hat{H}\ket{\Psi^2}\expval{\frac{d\Psi}{d\alpha}\frac{1}{\Psi}}
    \end{split}
\end{equation*}