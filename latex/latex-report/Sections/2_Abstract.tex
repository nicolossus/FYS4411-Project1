%================================================================
%------------------------- Abstract -----------------------------
%================================================================
\begin{abstract}
In this project, we build a variational Monte Carlo method to estimate the ground state energy of an ultracold, dilute Bose gas in harmonic traps. We use a trial wave function composed of a single particle Gaussian with a single variational parameter and a hard sphere Jastrow factor for pair correlations. We consider two Markov chain Monte Carlo sampling algorithms; a random walk Metropolis with Gaussian proposals and a Langevin Metropolis-Hastings with proposals driven according to the gradient flow of the probability density of the trial wave function. The methods are implemented in a Python framework with automatic differentiation, procedures for tuning scale parameters and gradient descent optimizations of the variational parameter. The blocking method, which accounts for the autocorrelation of a Markov chain, is used to calculate the statistical error of the variational energy. The implementation is verified by comparison with closed-form expressions available when we consider a system of non-interacting bosons only. We find that the Langevin Metropolis-Hastings has a more efficient sampling routine taken into account the extra computational costs compared with the random walk Metropolis. The implementation with automatic differentiation is only a small factor slower than the analytical implementation. We also find that the energies per particle of the interacting systems increase with the particle density. Our framework provides a solid foundation for building a more general variational Monte Carlo method that can handle more complex systems. 

%As for the energies of the harmonic oscillator systems examined, we found the ground state energies for the non-interacting systems to be $\expval{E}/N = \frac{3}{2}\hbar\omega_0$ for the spherical trap, $\expval{E}/N=2.414215\hbar\omega_0$ for the elliptical trap, and that the ground state energy per particle increased with the particle density for interacting systems.
% ... WHICH IS BETTER? RMW OR LMH? AD TIME VS ANALYTICAL? ETC...
% which is better, RWM or LMH? Statistical error estimations show that the LMH algorithm performs better**. The automatic differentiating sampler was only a small factor slower than our analytical sampler. 
% ** I feel like the standard error of the mean is highly dependent on our optimizer not being able to properly asses the gradients to high enough precision, due to the small number of samples in the batch. As the quality of the samples in the batch is higher, this is also an argument for the LMH algorithm? -- J. Dette er kanskje noe å nevne også? J-- Dette gir ikke mening i henhold til optimiseringssammenlignene i appendiks. 
\end{abstract}